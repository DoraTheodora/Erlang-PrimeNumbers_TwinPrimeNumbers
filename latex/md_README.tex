\subsubsection*{Concurrency Project}

\subparagraph*{Developer\+: Theodora Tataru }

\subparagraph*{Tutor\+: Joseph Kehoe }

\subparagraph*{Institute of Technology Carlow, 2021 }

\subsubsection*{Prime Numbers}

There are an infinite number of prime numbers. Amongst the primes there are what are known as twin primes.

A twin prime is a prime number that is either 2 less or 2 more than another prime number—for example, either member of the twin prime pair\+: {\ttfamily (41, 43)}. In other words, a twin prime is a prime that has a prime gap of two. Sometimes the term twin prime is used for a pair of twin primes; an alternative name for this is prime twin or prime pair.

It is (currently) unknown whether there are an infinite number of twin primes.

{\bfseries Examples include\+: } 
\begin{DoxyCode}
(3, 5), (5, 7), (11, 13), (17, 19), (29, 31), (41,43),... 
\end{DoxyCode}


The task is to write a parallel program that counts the number of primes less than n for any number n and also finds and lists all twin primes less than n. The code should run on linux.

{\bfseries Example output after running code with n = 50 would be\+: }


\begin{DoxyCode}
>
primeTwinCount 50

Total number of primes: 15

Twin Primes: 3, 5, 7, 11, 13, 17, 19, 29, 31, 41, 43

>
\end{DoxyCode}


{\bfseries The project consists of\+: }


\begin{DoxyItemize}
\item Source Code
\end{DoxyItemize}

{\bfseries Full source code includes\+: }
\begin{DoxyItemize}
\item R\+E\+A\+D\+ME
\item full installation instructions
\item Makefile
\item Doxygen
\item Configuration file
\item Report

A short report on the approach took to achieving maximum concurrency. This will contain\+: \begin{DoxyVerb}1. Pseudocode Outline of the algorithm illustrated with pseudocode;

2. Speedup Results Both absolute speedup and relative speedup should be calculated

3. Scalability Graph(s) showing the scalabiltiy of the code.
\end{DoxyVerb}


The report does not need to be long. It must be concise and on point. 
\end{DoxyItemize}